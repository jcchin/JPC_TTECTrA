% This file includes the content for the Introduction section

\lettrine[nindent=0pt]{V}{ariable} geometry components promise improvements
in gas turbine steady-state performance and safety. These improvements 
increase the complexity of the system with an additional actuation system. For
example, variable area fan nozzle (VAFN) can improve propulsive efficiency, reduce jet noise, increase fan 
surge margin, and control fan flutter margin. On the other hand, the VAFN 
actuator can reduce the efficiency increase due to its extra weight and 
the additional control variable - variable nozzle area - complicates the control 
system. One approach to reduce the complexity of the VAFN is to schedule the 
variable nozzle area as a function of the engine power level, but this approach does not provide the performance or 
safety benefit that can be gained by directly controlling the nozzle area.

The cost and benefit analysis for VAFN has considered only steady state analysis. What the VAFN area should be at different steady state points can be 
determined to realize the mentioned benefits. For example, top of climb requires 
a smaller VAFN area whereas take-off requires a larger VAFN area. However, the
gas turbine must switch from one steady state condition to another safely 
throughout the mission. During a transition from one steady state condition to 
another, slow VAFN response may cause thrust lapse or stall at different flight 
conditions.

The response time of VAFN depends strongly on actuator dynamics. Lighter 
actuators with shape memory alloys (SMA) \cite{Mabe:2008,Mabe:2008:Paris} 
were proposed instead of heavy and complex conventional hydraulic actuators to
reduce the fuel burn penalty due to actuator weight when VAFN is on-board. 
The SMA actuators are more flexible and lighter, however they respond slower 
than hydraulic actuators \cite{Rey:2001,Barooah:2002,Song:2007}. In addition, 
VAFN is a challenging application for SMA because above a certain temperature the
SMA material properties transition. Consequently, SMA actuator placement is 
important for VAFN application.

Before determining where to place the SMA actuator, a gas turbine controller 
must compensate for the slower SMA actuators while controlling either one 
variable (fuel flow rate with VAFN schedule) or two variables (fuel flow rate and 
VAFN). This study aims to evaluate how current SMA actuators perform with 
their dynamic characters in transient operation. If current SMA actuators do not 
meet the performance and safety requirements, the second aim is to determine 
how fast SMA dynamics must be to meet the requirements.

Based on small-scale test cases from Boeing \cite{Mabe:2008,Mabe:2008:Paris},
transient characteristics are also subject to various SMA configurations.
Using two-sets of monolithic Ni-40ti alloy actuators arranged antagonistically, 
nozzle area can be actively controlled in both expansion and contraction. This
approach reduces the complex mechanism required for wire-based SMA methods,
and also simplifies the balancing of SMA elements in a passively cooled
configuration. The ability to actively apply forces in opposing directions
allows more fine-grained control over actuator displacement given a broad range
of mass flows, velocities, and temperatures. SMA's configured in competing
directions also decouple actuator speed from the cooling rate of a single
actuator when operated in reverse. Given this configuration, a maximum of 20\%
change in VAFN area can be achieved on the order of 5-10 minutes, in either
direction, in a relatively underdamped system.

The paper is divided into three parts. First, the gas turbine engine model used in 
this study will be presented. Particular details about VAFN modeling will be given 
in addition to multi-design point cycle design values. After covering the gas 
turbine model, the controller development processes for scheduled and controlled
dynamic VAFN will be discussed. Finally, the results for the transient operation 
with the dynamic VAFN will be shown and analyzed.