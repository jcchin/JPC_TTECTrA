\lettrine[nindent=0pt]{V}{ariable} geometry components promise improvements in gas turbine performance and safety. However, the improvements come with more complexity in addition to a necessary trade-off study due to the additional actuator system. Fan variable area nozzle (FVAN) is an example for the improvements, increase in complexity, and trade-off. FVAN can improve propulsive efficiency, reduce jet noise, increase fan surge margin, and control fan flutter margin. On the other hand, the FVAN actuator system can reduce the efficiency increase due to its extra weight and the additional control variable - variable nozzle area - complicates the control system unless the variable nozzle area is scheduled. Scheduling does not provide better performance or safety than controlling directly. 

The cost and benefit analysis for FVAN has considered only steady state analysis so far. What the FVAN area should be at different steady state points can be determined to realize the mentioned benefits. For example, top of climb requires a smaller FVAN area whereas take-off requires a larger FVAN area. However, the gas turbine must switch from one steady state condition to another safely throughout the mission. During a transition from one steady state condition to another, FVAN must respond in concert with shaft dynamics for performance and safety. If FVAN doesn't respond in accordance with shaft dynamics, either thrust lapse or stall can occur at different flight conditions. 

How fast FVAN responds, depends strongly on how fast the actuator is. Lighter actuators with shape memory alloys (SMA) \cite{Mabe:2008,Mabe:2008:Paris} were proposed instead of heavy and complex conventional hydraulic actuators to reduce the fuel burn penalty due to actuator weight when FVAN is on-board. The SMA actuators are more flexible, lighter and slower than hydraulic actuators \cite{Rey:2001,Barooah:2002,Song:2007}. In addition, FVAN is a challenging application for SMA because above a certain temperature SMA material properties transition. Consequently, SMA actuator placement is important for FVAN application.  

Before determining where to place the SMA actuator, a gas turbine controller must compensate for the slower SMA actuators while controlling either one variable (fuel flow rate with FVAN schedule) or two variables (fuel flow rate and FVAN). This study aims to evaluate how current SMA actuators perform with their dynamic characters in transient operation. If current SMA actuators do not meet the performance and safety requirements, the second aim is to determine how fast SMA dynamics must be to meet the requirements.

Based on small-scale test cases from Boeing \cite{Mabe:2008,Mabe:2008:Paris},
transients characteristics are also subject to various SMA configurations.
Using two-sets of monolithic Ni-40ti alloy actuators arranged antagonistically,
nozzle area can be actively controlled in both expansion and contraction. This
approach reduces the complex mechanism required for wire-based SMA methods,
and also simplifies the balancing of SMA elements in a passively cooled
configuration. The ability to actively apply forces in opposing directions
allows more fine-grained control over actuator displacement given a broad range
of mass flows, velocities, and temperatures. SMA's configured in competing
directions also decouple actuator speed from the cooling rate of a single
actuator when operated in reverse. Given this configuration, a maximum of 20\%
change in FVAN area can be achieved on the order of 5-10 minutes, in either
direction, in a relatively underdamped system.

The paper is divided into three parts. First, the gas turbine engine model used in this study will be presented. Particular details about FVAN modeling will be given in addition to multi-design point cycle design values. After covering the gas turbine model, the controller development processes for scheduled and controlled dynamic FVAN will be discussed. Finally, the results for the transient operation with the dynamic FVAN will be shown and analyzed. 