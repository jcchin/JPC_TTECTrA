% This file contains the content for the Abstract section.

Variable area fan nozzles (VAFN) can provide several benefits, such as improving
propulsive efficiency, reducing jet noise, increasing fan surge margin, and 
controlling fan flutter margin, by changing the VAFN area as a function of the 
current operating condition.  For example, the top of climb flight condition 
requires a smaller VAFN area whereas take-off requires a larger VAFN area. 
When analyzing the performance at these steady-state flight conditions, 
the aforementioned benefits are observed.  However, the gas turbine must 
operate one steady-state condition to another safely throughout the mission to 
realize these benefits. 

During a transition from one steady-state condition to another, slow VAFN 
response may cause thrust lapse or stall at different flight conditions. The 
response time of VAFN depends strongly on actuator dynamics. Lighter actuators
with shape memory alloys (SMA) were proposed instead of heavy and complex 
conventional hydraulic actuators to reduce the fuel burn penalty due to actuator
weight when VAFN is on-board. The SMA actuators are more flexible and lighter, 
but also respond slower than the hydraulic actuators. Therefore, a gas turbine 
controller must compensate for the slower SMA actuators while controlling either 
one variable (fuel flow rate with VAFN schedule) or two variables (fuel flow rate 
and VAFN). Technology impact evaluation considers steady state performance 
analysis. However, some new technologies require transient analysis and 
controller design. This study aims to demonstrate the interactions between 
transient modeling, control design, and technology using the case study for 
current SMA actuators performance.

The SMA actuators will be represented by a first order lag with representative 
time constant values selected from previous work on SMA actuators. Using a gas
turbine model in the small single aisle thrust class, closed-loop controllers will be
developed for the cases where VAFN runs with a schedule and when a 
closed-loop controller directly regulates the VAFN area. Gas turbine transient 
performance for the two VAFN cases will be analyzed based on thrust lapse and 
stall margin. A signal with a series of steps, chops, and ramps will be used to 
test the two developed controllers. If the gas turbine cannot reach 98\% of 
maximum thrust in less than five seconds (FAA requirement) or keep a safe stall 
margin throughout the transient operation, the time constant will be reduced to 
determine the SMA actuator response required to meet the performance and 
safety requirements.