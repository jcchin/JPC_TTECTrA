Fan variable area nozzles (FVAN) provide several benefits such as improving propulsive efficiency, reducing jet noise, increasing fan surge margin, and controlling fan flutter margin. Changing FVAN area for different steady state conditions provide the listed benefits. For example, top of climb requires a smaller FVAN area whereas take-off requires a larger FVAN area. When the performance is analyzed at these flight conditions, the mentioned benefits are observed. However, the gas turbine must switch from one steady state condition to another safely throughout the mission to realize these benefits. 

During a transition from one steady state condition to another, FVAN must respond in concert with shaft dynamics for performance and safety. If FVAN doesn't respond in accordance with shaft dynamics, either thrust lapse or stall can occur at different flight conditions. How fast FVAN responds, depends strongly on how fast the actuator is. Lighter actuators with shape memory alloys (SMA) were proposed instead of heavy and complex conventional hydraulic actuators to reduce the fuel burn penalty due to actuator weight when FVAN is on-board. The SMA actuators are more flexible, lighter and slower than hydraulic actuators. Therefore, a gas turbine controller must compensate for the slower SMA actuators while controlling either one variable (fuel flow rate with FVAN schedule) or two variables (fuel flow rate and FVAN). This study aims to evaluate how current SMA actuators perform. If current SMA actuators fail, how slow SMA actuators are acceptable will be determined.

A first order lag will represent the SMA actuators and the representative time constant values are selected from the literature on the SMA actuators. Using a gas turbine model in small single aisle thrust class, controllers will be developed for the cases where FVAN runs with a schedule or a controller changes FVAN area directly. Gas turbine transient performance for the two FVAN cases will be analyzed based on thrust lapse and stall margin. A signal with a series of steps, chops, and ramps will be used to test the two developed controllers. If the gas turbine cannot reach 98\% of maximum thrust in less than five seconds (FAA requirement) or keep a safe stall margin throughout the transient operation, the time constant will be reduced to observe how fast the SMA actuator needs to be for performance and safety.