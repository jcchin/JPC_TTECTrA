\documentclass[]{aiaa-tc}% insert '[draft]' option to show overfull boxes

\usepackage{mathtools,amssymb}

\usepackage{gensymb}

\usepackage{array,multirow,rotating}
\usepackage{ifthen}
\usepackage{graphicx}
\usepackage{varioref}%  smart page, figure, table, and equation referencing
\usepackage{wrapfig}%   wrap figures/tables in text (i.e., Di Vinci style)
\usepackage{threeparttable}% tables with footnotes
\usepackage{dcolumn}%   decimal-aligned tabular math columns
 \newcolumntype{d}{D{.}{.}{-1}}
\usepackage{nomencl}%   nomenclature generation via makeindex
\usepackage{makeidx}
 \makenomenclature
\usepackage{subfigure}% subcaptions for subfigures
\usepackage{subfigmat}% matrices of similar subfigures, aka small mulitples
\usepackage{fancyvrb}%  extended verbatim environments
 \fvset{fontsize=\footnotesize,xleftmargin=2em}
\usepackage{lettrine}%  dropped capital letter at beginning of paragraph
%\usepackage[dvips]{dropping}% alternative dropped capital package
\usepackage{hyperref}%  hyperlinks [must be loaded after dropping]

\hypersetup{
  colorlinks   = true, %Colours links instead of ugly boxes
  urlcolor     = black, %Colour for external hyperlinks
  linkcolor    = black, %Colour of internal links
  citecolor   = black %Colour of citations
}

\graphicspath{ {./images/} }

\title{AIAA Latex Template}

\author{
 Metin F. Ozcan\thanks{Graduate Research Assistant, Aerospace Systems Design Laboratory (ASDL), 275 Ferst Drive NW, Atlanta, GA 30332-0150, and AIAA Student Member.}
 \ , Imon Chakraborty\footnotemark[1]
 \ , and Dimitri N. Mavris\thanks{Boeing Regents Professor of Advanced Aerospace Systems Analysis, Director at ASDL, and AIAA Fellow.}  \\
 {\normalsize\itshape
  School of Aerospace Engineering, Georgia Institute of Technology, Atlanta, GA, 30332}
}

% Data used by 'handcarry' option
\AIAApapernumber{YEAR-NUMBER}
\AIAAconference{Conference Name, Date, and Location}
\AIAAcopyright{\AIAAcopyrightD{YEAR}}

% Define commands to assure consistent treatment throughout document
\newcommand{\eqnref}[1]{(\ref{#1})}
\newcommand{\class}[1]{\texttt{#1}}
\newcommand{\package}[1]{\texttt{#1}}
\newcommand{\file}[1]{\texttt{#1}}
\newcommand{\BibTeX}{\textsc{Bib}\TeX}

\newcommand{\superscript}[1]{\ensuremath{^{\textnormal{#1}}}}
\newcommand{\subscript}[1]{\ensuremath{_{\textnormal{#1}}}}

%\RequirePackage{ifthen}
%\renewcommand{\nomgroup}[1]{%
%\ifthenelse{\equal{#1}{G}}{\item[\textbf{Greek Symbols}]}{
%\ifthenelse{\equal{#1}{S}}{\item[\textbf{Subscripts}]}{}}}

\begin{document}

\nomenclature{$TONE$}{The only nomenclature entry, Units go here}

\maketitle

\begin{abstract}

% This file contains the content for the Abstract section.

Variable area fan nozzles (VAFN) can provide several benefits, such as improving
propulsive efficiency, reducing jet noise, increasing fan surge margin, and 
controlling fan flutter margin, by changing the VAFN area as a function of the 
current operating condition.  For example, the top of climb flight condition 
requires a smaller VAFN area whereas take-off requires a larger VAFN area. 
When analyzing the performance at these steady-state flight conditions, 
the aforementioned benefits are observed.  However, the gas turbine must 
operate one steady-state condition to another safely throughout the mission to 
realize these benefits. 

During a transition from one steady-state condition to another, slow VAFN 
response may cause thrust lapse or stall at different flight conditions. The 
response time of VAFN depends strongly on actuator dynamics. Lighter actuators
with shape memory alloys (SMA) were proposed instead of heavy and complex 
conventional hydraulic actuators to reduce the fuel burn penalty due to actuator
weight when VAFN is on-board. The SMA actuators are more flexible and lighter, 
but also respond slower than the hydraulic actuators. Therefore, a gas turbine 
controller must compensate for the slower SMA actuators while controlling either 
one variable (fuel flow rate with VAFN schedule) or two variables (fuel flow rate 
and VAFN). Technology impact evaluation considers steady state performance 
analysis. However, some new technologies require transient analysis and 
controller design. This study aims to demonstrate the interactions between 
transient modeling, control design, and technology using the case study for 
current SMA actuators performance.

The SMA actuators will be represented by a first order lag with representative 
time constant values selected from previous work on SMA actuators. Using a gas
turbine model in the small single aisle thrust class, closed-loop controllers will be
developed for the cases where VAFN runs with a schedule and when a 
closed-loop controller directly regulates the VAFN area. Gas turbine transient 
performance for the two VAFN cases will be analyzed based on thrust lapse and 
stall margin. A signal with a series of steps, chops, and ramps will be used to 
test the two developed controllers. If the gas turbine cannot reach 98\% of 
maximum thrust in less than five seconds (FAA requirement) or keep a safe stall 
margin throughout the transient operation, the time constant will be reduced to 
determine the SMA actuator response required to meet the performance and 
safety requirements.

\end{abstract}

\printnomenclature % creates nomenclature section produced by MakeIndex

\section{Introduction}

% This file includes the content for the Introduction section

\lettrine[nindent=0pt]{V}{ariable} geometry components promise improvements
in gas turbine steady-state performance and safety. These improvements 
increase the complexity of the system with an additional actuation system. For
example, variable area fan nozzle (VAFN) can improve propulsive efficiency, reduce jet noise, increase fan 
surge margin, and control fan flutter margin. On the other hand, the VAFN 
actuator can reduce the efficiency increase due to its extra weight and 
the additional control variable - variable nozzle area - complicates the control 
system. One approach to reduce the complexity of the VAFN is to schedule the 
variable nozzle area as a function of the engine power level, but this approach does not provide the performance or 
safety benefit that can be gained by directly controlling the nozzle area.

The cost and benefit analysis for VAFN has considered only steady state analysis. What the VAFN area should be at different steady state points can be 
determined to realize the mentioned benefits. For example, top of climb requires 
a smaller VAFN area whereas take-off requires a larger VAFN area. However, the
gas turbine must switch from one steady state condition to another safely 
throughout the mission. During a transition from one steady state condition to 
another, slow VAFN response may cause thrust lapse or stall at different flight 
conditions.

The response time of VAFN depends strongly on actuator dynamics. Lighter 
actuators with shape memory alloys (SMA) \cite{Mabe:2008,Mabe:2008:Paris} 
were proposed instead of heavy and complex conventional hydraulic actuators to
reduce the fuel burn penalty due to actuator weight when VAFN is on-board. 
The SMA actuators are more flexible and lighter, however they respond slower 
than hydraulic actuators \cite{Rey:2001,Barooah:2002,Song:2007}. In addition, 
VAFN is a challenging application for SMA because above a certain temperature the
SMA material properties transition. Consequently, SMA actuator placement is 
important for VAFN application.

Before determining where to place the SMA actuator, a gas turbine controller 
must compensate for the slower SMA actuators while controlling either one 
variable (fuel flow rate with VAFN schedule) or two variables (fuel flow rate and 
VAFN). This study aims to evaluate how current SMA actuators perform with 
their dynamic characters in transient operation. If current SMA actuators do not 
meet the performance and safety requirements, the second aim is to determine 
how fast SMA dynamics must be to meet the requirements.

Based on small-scale test cases from Boeing \cite{Mabe:2008,Mabe:2008:Paris},
transient characteristics are also subject to various SMA configurations.
Using two-sets of monolithic Ni-40ti alloy actuators arranged antagonistically, 
nozzle area can be actively controlled in both expansion and contraction. This
approach reduces the complex mechanism required for wire-based SMA methods,
and also simplifies the balancing of SMA elements in a passively cooled
configuration. The ability to actively apply forces in opposing directions
allows more fine-grained control over actuator displacement given a broad range
of mass flows, velocities, and temperatures. SMA's configured in competing
directions also decouple actuator speed from the cooling rate of a single
actuator when operated in reverse. Given this configuration, a maximum of 20\%
change in VAFN area can be achieved on the order of 5-10 minutes, in either
direction, in a relatively underdamped system.

The paper is divided into three parts. First, the gas turbine engine model used in 
this study will be presented. Particular details about VAFN modeling will be given 
in addition to multi-design point cycle design values. After covering the gas 
turbine model, the controller development processes for scheduled and controlled
dynamic VAFN will be discussed. Finally, the results for the transient operation 
with the dynamic VAFN will be shown and analyzed.

\section{The section that follows the introduction}
Content for this section goes here.

% Two figures side by side
\begin{figure}[!htb]
\begin{minipage}[b]{0.5\linewidth}
\begin{center}
\includegraphics[width=\textwidth,height=\textheight,keepaspectratio]{placeholder.png}
\end{center}
\end{minipage}
\hspace{0.5cm}
\begin{minipage}[b]{0.5\linewidth}
\begin{center}
\includegraphics[width=\textwidth,height=\textheight,keepaspectratio]{placeholder.png}
\end{center}
\end{minipage}
\caption{Two placeholder figures side by side}
\label{fig:side_by_side_placeholders}
\end{figure}

\section{Just another section}

\autoref{eq:example_equation} is the first equation in this document.

% Example equation
\begin{equation} \label{eq:example_equation}
p(z)\approx \frac{k}{NV}
\end{equation}

You can also write down a group of equations as in \autoref{eq:example_eqn_group}.

% Example equation array
\begin{eqnarray}
\label{eq:example_eqn_group}
c_{D} = \frac{2\pi^{D/2}}{D \cdot \Gamma(D/2)} \\
\Gamma(t) = \int_{0}^{\infty} x^{t} \mathrm{e}^{-x} \frac{\mathrm{d}x}{x} \nonumber
\end{eqnarray}

\section{A section with a table}

After introducing figures and equations, it is time to introduce a table as in \autoref{tbl:example}.

% Example table
\begin{table}[!htb]
\caption{Example table}
\label{tbl:example}
\begin{center}
\begin{tabular}{|c|c|}
\hline
\textbf{First column} & \textbf{Second column} \\ \hline
First item & 1.5 \\ \hline
Second item & 0.89 \\ \hline
Third item & 2650 \\ \hline
\end{tabular}
\end{center}
\end{table}

The simple table in \autoref{tbl:example} is followed by a more involved example in \autoref{tbl:complex_example}.

\begin{table}[!htb]
\caption{Evaluation results for the input signals selected from the literature}
\label{tbl:complex_example}
\begin{center}
\begin{tabular}{|c|c|c|c|c|}
\cline{2-5}
\multicolumn{1}{c|}{} & \multicolumn{4}{c|}{Density Estimation Evaluation Points} \\
\cline{2-5}
\multicolumn{1}{c|}{} & Boundary & Interior & Equilibrium & Overall \\ \hline
Steps with multisine & 0.0577 & 0.0398 & 0.0206 & 0.0456 \\ \hline
Triangular wave & 0.1525 & 0.0751 & 0.1069 & 0.1009 \\ \hline
Step and chop series & 0.0063 & 0.0564 & 0.0464 & 0.0397 \\ \hline
\end{tabular}
\end{center}
\end{table}

\section{Conclusion}

% This file contains the contents for the Conclusion section. It also includes  future work ideas.

In the conclusions section we will reiterate if the current SMA actuator 
performance is acceptable with or without a schedule. We will mention again 
whether they cause unacceptable thrust lapse or fan stall margin. If not, we will 
reemphasize how fast SMA actuators need to be for performance and safety.

For future work, the analysis can be redone for the two cases in this study when
SMA actuators are modeled with a higher fidelity instead of a first order lag. 
Also, transient performance with an SMA actuator can be compared against a 
hydraulic actuator. When reliable weight estimations for SMA and hydraulic 
actuators are available, how the actuators satisfying transient performance and
safety requirements affect aircraft mission performance due to additional weight
can be studies for different size vehicles.

% produces the bibliography section when processed by BibTeX
\bibliography{reference_database}
\bibliographystyle{aiaa}

\end{document} 